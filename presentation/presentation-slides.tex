\documentclass{beamer}
\usepackage[utf8]{inputenc}
\usepackage[french]{babel}
\usepackage{listings}
\usepackage{tikz}
\usetikzlibrary{shapes.geometric, arrows.meta, positioning, fit, backgrounds}
\usetheme{Madrid}
\usecolortheme{whale}

% Configuration Lisp pour les slides
\lstdefinelanguage{Lisp}{
  morekeywords={defstruct, defun, loop, when, push},
  basicstyle=\ttfamily\scriptsize,
  keywordstyle=\color{blue}\bfseries,
  commentstyle=\color{green!60!black},
  stringstyle=\color{orange},
}

\title{Système Expert d'Aide au Diagnostic Orthodontique}
\subtitle{TP02 - Intelligence Artificielle Symbolique}
\author{Loudiern Tharon \and Lou Aubert-Debrue}
\institute{Université de technologie de Compiègne}
\date{19 Décembre 2025}

\begin{document}

\begin{frame}
  \titlepage
\end{frame}

\begin{frame}{Plan de la présentation}
  \tableofcontents
\end{frame}

\section{Contexte et Objectifs}
\begin{frame}{Problématique Médicale}
  \textbf{Domaine :} Orthopédie Dento-Faciale (ODF).
  \vspace{0.5cm}
  
  \textbf{Le Problème :}
  \begin{itemize}
      \item Complexité du diagnostic (multiples paramètres : squelettique, dentaire, fonctionnel).
      \item Risque d'erreur dans le choix de l'appareillage (fixe vs fonctionnel vs chirurgie).
  \end{itemize}
  
  \vspace{0.5cm}
  \textbf{Notre Solution :}
  \begin{itemize}
      \item Système Expert d'Ordre 0+.
      \item Aide à la décision pour praticiens généralistes ou étudiants.
  \end{itemize}
\end{frame}

\section{Base de Connaissances}
\begin{frame}{Modélisation des Connaissances}
  Sources : Cochrane Reviews, HAS, Classification d'Angle.
  
  \begin{block}{Exemple de Règle Métier (R-A2)}
    \textbf{SI} Relation Molaire = Classe 2 \\
    \textbf{ET} Incisives = Proclinées \\
    \textbf{ET} Overjet > 5mm \\
    \textbf{ALORS} Diagnostic = Classe II Division 1
  \end{block}

  \begin{block}{Exemple de Règle Thérapeutique (R-B1)}
    \textbf{SI} Diagnostic = Classe II Div 1 \\
    \textbf{ET} 8 $\le$ Âge $\le$ 12 ans \\
    \textbf{ET} Coopération = Bonne \\
    \textbf{ALORS} Appareil = Fonctionnel (Activateur)
  \end{block}
\end{frame}

\begin{frame}{Arbre de Décision du Système Expert}
  \begin{center}
  \begin{tikzpicture}[
    scale=0.65, transform shape,
    node distance=0.8cm and 0.5cm,
    input/.style={rectangle, rounded corners, draw=blue!70, fill=blue!15, text width=2cm, align=center, font=\scriptsize\bfseries, minimum height=0.6cm},
    diag/.style={rectangle, draw=orange!70, fill=orange!20, text width=2.2cm, align=center, font=\scriptsize\bfseries, minimum height=0.6cm},
    app/.style={rectangle, draw=green!70, fill=green!20, text width=2cm, align=center, font=\scriptsize\bfseries, minimum height=0.6cm},
    ci/.style={rectangle, draw=red!70, fill=red!20, text width=2cm, align=center, font=\scriptsize\bfseries, minimum height=0.6cm},
    arrow/.style={-{Stealth[scale=0.8]}, thick}
  ]
    % Niveau 0 : Entrées utilisateur
    \node[input] (age) {Âge};
    \node[input, right=of age] (mol) {Relation\\Molaire};
    \node[input, right=of mol] (anb) {ANB/Wits};
    \node[input, right=of anb] (ov) {Overjet\\Overbite};
    \node[input, right=of ov] (inc) {Incisives};
    \node[input, right=of inc] (coop) {Coopération};
    
    % Niveau 1 : Diagnostics (Module A)
    \node[diag, below=1.2cm of mol] (cl1) {Classe I\\Mineure};
    \node[diag, right=0.4cm of cl1] (cl2d1) {Classe II\\Div 1};
    \node[diag, right=0.4cm of cl2d1] (cl2d2) {Classe II\\Div 2};
    \node[diag, right=0.4cm of cl2d2] (cl3) {Classe III\\Squel.};
    
    % Niveau 2 : Appareils (Module B)
    \node[app, below=1.2cm of cl1] (align) {Aligneurs};
    \node[app, right=0.4cm of align] (fonc) {Fonctionnel};
    \node[app, right=0.4cm of fonc] (hg) {Headgear};
    \node[app, right=0.4cm of hg] (masque) {Masque\\Delaire};
    
    % Niveau 3 : CI/Chirurgie (Module C)
    \node[ci, below=1.2cm of fonc] (ci) {Contre-\\Indiqué};
    \node[ci, right=1cm of ci] (chir) {Chirurgie};
    
    % Flèches
    \draw[arrow, blue!60] (mol) -- (cl1);
    \draw[arrow, blue!60] (mol) -- (cl2d1);
    \draw[arrow, blue!60] (anb) -- (cl3);
    \draw[arrow, blue!60] (ov) -- (cl2d1);
    \draw[arrow, blue!60] (inc) -- (cl2d2);
    
    \draw[arrow, orange!60] (cl1) -- (align);
    \draw[arrow, orange!60] (cl2d1) -- (fonc);
    \draw[arrow, orange!60] (cl2d1) -- (hg);
    \draw[arrow, orange!60] (cl3) -- (masque);
    
    \draw[arrow, red!60] (cl3) -- (chir);
    
    % Labels des modules
    \node[left=0.3cm of age, font=\tiny\itshape, text=blue!70] {ENTRÉES};
    \node[left=0.3cm of cl1, font=\tiny\itshape, text=orange!70] {MODULE A};
    \node[left=0.3cm of align, font=\tiny\itshape, text=green!70] {MODULE B};
    \node[left=0.3cm of ci, font=\tiny\itshape, text=red!70] {MODULE C};
  \end{tikzpicture}
  \end{center}
  
  \vspace{-0.3cm}
  \begin{columns}[T]
    \column{0.5\textwidth}
    \footnotesize
    \textcolor{blue!70}{\textbf{Entrées :}} Données patient (utilisateur)\\
    \textcolor{orange!70}{\textbf{Module A :}} Règles diagnostiques
    
    \column{0.5\textwidth}
    \footnotesize
    \textcolor{green!70}{\textbf{Module B :}} Choix appareillage\\
    \textcolor{red!70}{\textbf{Module C :}} Contre-indications
  \end{columns}
\end{frame}

\begin{frame}[fragile]{Code : Structure des Règles (Module A)}
  \textbf{Exemple R-A2 : Diagnostic Classe II Division 1}
  \begin{lstlisting}[language=Lisp]
(push (make-regle 
       :id 'R-A2 
       :premisses '((relation-molaire equal classe-2)
                    (incisives-max equal proclinees)
                    (overjet > 5))
       :conclusions '((diagnostic classe-2-div-1))
       :cf 0.90 
       :description "Classe II division 1") 
      *base-regles*)
  \end{lstlisting}
  
  \begin{alertblock}{Format des prémisses}
    \texttt{(attribut operateur valeur)} $\to$ Triplet Ordre 0+\\
    Opérateurs : \texttt{equal}, \texttt{>}, \texttt{<}, \texttt{>=}, \texttt{<=}, \texttt{member}
  \end{alertblock}
\end{frame}

\begin{frame}[fragile]{Code : Évaluation des Conditions}
  \textbf{Fonction clé : \texttt{evaluer-condition}}
  \begin{lstlisting}[language=Lisp]
(defun evaluer-condition (condition)
  (let* ((attr (first condition))
         (op   (second condition))
         (val-ref (third condition))
         (fait-trouve (valeur-fait attr)))
    (if (null fait-trouve)
        :inconnu  ;; Fait non renseigne
        (case op
          (equal  (equal val-reelle val-ref))
          (>      (> val-reelle val-ref))
          (<      (< val-reelle val-ref))
          (member (member val-reelle val-ref))))))
  \end{lstlisting}
  
  \begin{block}{Justification Ordre 0+}
    $\bullet$ 6 opérateurs de comparaison (au-delà du booléen pur)\\
    $\bullet$ Retour \texttt{:inconnu} $\neq$ \texttt{nil} (distinction faux/non-renseigné)
  \end{block}
\end{frame}

\section{Implémentation Technique}
\begin{frame}[fragile]{Choix Techniques en Common Lisp}
  \textbf{Représentation :} Structures (`defstruct`) pour une meilleure sémantique.
  
  \begin{lstlisting}[language=Lisp]
(defstruct fait
  attribut   ; ex: 'overjet
  valeur     ; ex: 6
  source)    ; :utilisateur ou :deduit

(defstruct regle
  id premisses conclusions active)
  \end{lstlisting}
  
  \vspace{0.5cm}
  \textbf{Moteur d'Inférence :}
  \begin{itemize}
      \item \textbf{Chaînage Avant} (Forward Chaining).
      \item Méthode par \textit{saturation} de la base de faits.
      \item Gestion des comparateurs : $>, <, \ge, =, \text{member}$.
  \end{itemize}
\end{frame}

\begin{frame}[fragile]{Logique du Moteur}
  \begin{lstlisting}[language=Lisp]
(defun chainage-avant ()
  (loop while nouveau-fait-trouve do
    (dolist (r *base-regles*)
      (when (regle-active r)
        (let ((ok (verifier-premisses r)))
          (when ok
            (declencher-regle r)
            (desactiver-regle r)))))))
  \end{lstlisting}
  
  \begin{alertblock}{Pourquoi le chaînage avant ?}
    En médecine, on part des symptômes (données) pour aller vers le diagnostic (but). C'est une approche "Data-Driven".
  \end{alertblock}
\end{frame}

\section{Démonstration et Résultats}
\begin{frame}{Scénario de Test : L'enfant "Classe II"}
  \textbf{Données Patient :}
  \begin{itemize}
      \item Âge : 10 ans
      \item Molaire : Classe 2, Overjet : 6mm
      \item Coopération : Bonne
  \end{itemize}

  \vspace{0.5cm}
  \textbf{Exécution du Système :}
  \begin{enumerate}
      \item \texttt{Cycle 1} : Règle R-A2 s'active $\to$ \textbf{Diag : Classe II Div 1}.
      \item \texttt{Cycle 2} : Règle R-B1 s'active (grâce au diag + âge) $\to$ \textbf{Appareil : Fonctionnel}.
      \item \texttt{Cycle 3} : Aucune nouvelle déduction. Arrêt.
  \end{enumerate}
  
  \textbf{Validation :} Cohérent avec les revues systématiques (Koretsi et al., 2015).
\end{frame}

\section{Collaboration IA}
\begin{frame}{Usage de l'IA Générative}
  \textbf{Rôles définis :}
  \begin{itemize}
      \item \textbf{Humain :} Expert du domaine (fournit les règles, valide la logique médicale).
      \item \textbf{IA (Gemini) :} Expert technique (implémentation Lisp, syntaxe, tests unitaires).
  \end{itemize}

  \vspace{0.5cm}
  \textbf{Apports clés :}
  \begin{itemize}
      \item Génération rapide de la structure du code (`boilerplate`).
      \item Création de jeux de tests automatisés (`test-edge-cases`).
      \item Débogage des parenthèses Lisp.
  \end{itemize}
\end{frame}

\section{Glossaire}
\begin{frame}{Glossaire Technique - Orthodontie}
  \begin{description}
      \item[Classes d'Angle] Classification squeletto-dentaire fondamentale. Classe I (normal), II (rétrognathe), III (prognathe). Base du diagnostic (Règles R-A1 à R-A5).
      
      \item[ANB] Angle céphalométrique (points A-N-B). Normal : $2$--$4^\circ$. ANB $<0^\circ$ = Classe III squelettique (Règle R-A4).
      
      \item[Overjet] Distance horizontale incisives sup./inf. Normal : 2-3mm. Overjet $>5$mm = Classe II Div 1 (R-A2).
      
      \item[Masque de Delaire] Traction faciale pour Classe III jeune (6-9 ans). Stimule la croissance maxillaire (R-B3).
  \end{description}
\end{frame}

\begin{frame}{Glossaire Technique - IA Symbolique}
  \begin{description}
      \item[Système Expert Ordre 0+] Logique propositionnelle + prédicats ($>, <, =$). Pas de variables ni quantificateurs.
      
      \item[Chaînage Avant] Inférence dirigée par les \textbf{données}. Part des symptômes vers le diagnostic. Implémenté dans \texttt{chainage-avant()} (ligne 200).
      
      \item[Saturation] Application itérative des règles jusqu'à absence de nouveaux faits. Garantie de terminaison via flag \texttt{active}.
      
      \item[Base de Faits] Mémoire de travail dynamique (\texttt{*base-faits*}). Contient les connaissances avérées sur le patient.
  \end{description}
\end{frame}

\begin{frame}{Glossaire Technique - Implémentation Lisp}
  \begin{description}
      \item[defstruct] Création de types structurés. \texttt{fait} (attribut/valeur/source) et \texttt{regle} (prémisses/conclusions). Génère accesseurs automatiques.
      
      \item[evaluer-condition] Fonction vérifiant si une condition est vraie (ligne 183). Gère 6 opérateurs : \texttt{equal}, $>$, $<$, $\ge$, $\le$, \texttt{member}.
      
      \item[Flag active] Attribut booléen dans \texttt{regle}. Désactivé après déclenchement pour éviter boucles infinies (ligne 223).
      
      \item[Tests Unitaires] Fonctions automatisées : \texttt{test-edge-cases}, \texttt{test-integration-complet}. Score de réussite calculé.
  \end{description}
\end{frame}

\section{Conclusion}
\begin{frame}{Conclusion et Perspectives}
  \textbf{Bilan :}
  \begin{itemize}
      \item Système fonctionnel et robuste (gestion des erreurs).
      \item Base de règles validée par la bibliographie.
      \item Code modulaire et documenté.
  \end{itemize}

  \vspace{0.5cm}
  \textbf{Limites (Ordre 0+) :}
  \begin{itemize}
      \item Effets de seuil (ex: âge strict à 12 ans).
      \item Manque de nuance (pas de "peut-être").
  \end{itemize}

  \vspace{0.5cm}
  \begin{center}
      \textit{Merci de votre attention.}
  \end{center}
\end{frame}

\end{document}
