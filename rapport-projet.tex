\documentclass[a4paper,12pt]{report}
\usepackage[utf8]{inputenc}
\usepackage[T1]{fontenc}
\usepackage[french]{babel}
\usepackage{geometry}
\usepackage{graphicx}
\usepackage{listings}
\usepackage{xcolor}
\usepackage{hyperref}
\usepackage{csquotes}
\usepackage[style=apa, backend=biber]{biblatex}

\addbibresource{references.bib}

\geometry{hmargin=2.5cm,vmargin=2.5cm}

% Configuration pour le code Lisp
\definecolor{codegreen}{rgb}{0,0.6,0}
\definecolor{codegray}{rgb}{0.5,0.5,0.5}
\definecolor{codepurple}{rgb}{0.58,0,0.82}
\definecolor{backcolour}{rgb}{0.95,0.95,0.92}

\lstdefinelanguage{Lisp}{
  morekeywords={defun, defparameter, defstruct, let, loop, cond, if, dolist, push, when, unless, setf, format, equal, case, make-regle, make-fait},
  sensitive=true,
  morecomment=[l]{;},
  morestring=[b]",
}

\lstset{
    language=Lisp,
    backgroundcolor=\color{backcolour},   
    commentstyle=\color{codegreen},
    keywordstyle=\color{codepurple},
    numberstyle=\tiny\color{codegray},
    stringstyle=\color{blue},
    basicstyle=\ttfamily\footnotesize,
    breakatwhitespace=false,         
    breaklines=true,                 
    captionpos=b,                    
    keepspaces=true,                 
    numbers=left,                    
    numbersep=5pt,                  
    showspaces=false,                
    showstringspaces=false,
    showtabs=false,                  
    tabsize=2
}

\title{
    \textbf{Conduite d'expertise d'un Système Expert d'Ordre 0+}\\
    \large Aide à la décision en Orthopédie Dento-Faciale\\
    \small TP02 - Intelligence Artificielle Symbolique
}
\author{Étudiant(e) XYZ \and Étudiant(e) ABC}
\date{Janvier 2026}

\begin{document}

\maketitle
\tableofcontents
\newpage

\section{Introduction}

\subsection{Contexte et Problématique}
L'Orthopédie Dento-Faciale (ODF), communément appelée orthodontie, est une discipline complexe nécessitant l'intégration de multiples paramètres cliniques (squelettiques, dentaires, esthétiques) pour établir un diagnostic et un plan de traitement. 

La problématique traitée dans ce projet est la création d'un système d'aide à la décision pour la \textbf{sélection du type d'appareil orthodontique} chez l'enfant et l'adulte. Face à la multitude de dispositifs existants (fonctionnels, fixes, aligneurs, etc.), le risque d'erreur de diagnostic ou de mauvaise indication thérapeutique est réel pour un praticien débutant.

\subsection{Justification du Système Expert d'Ordre 0+}
Le choix d'un système expert d'ordre 0+ (logique propositionnelle étendue aux prédicats simples, sans gestion d'incertitude probabiliste type Bayes) se justifie par la nature des classifications orthodontiques :
\begin{itemize}
    \item \textbf{Déterminisme des classifications :} Les classes d'Angle (I, II, III) sont définies par des critères géométriques stricts \parencite{angle1899}.
    \item \textbf{Protocoles standardisés :} Les recommandations de la Haute Autorité de Santé \parencite{has2002} fournissent des arbres de décision clairs (ex: Si ANB < 0 alors Classe III).
    \item \textbf{Absence de temps réel :} Le diagnostic se fait sur des données statiques (radiographies, mesures cliniques) à un instant $t$.
\end{itemize}

Ce système permet donc de formaliser le raisonnement clinique de manière explicite et vérifiable.

\section{Base de Connaissances}

\subsection{Formalisation des Règles}
La base de connaissances a été construite à partir d'une analyse bibliographique rigoureuse. Elle est divisée en trois modules : Diagnostic, Appareillage, et Contre-indications.

Voici quelques exemples de règles formalisées (SI ... ALORS ...) :

\begin{itemize}
    \item \textbf{Règle R-A2 (Diagnostic Classe II)} :
    SI \textit{Relation Molaire} = Classe 2 ET \textit{Incisives Maxillaires} = proclinées ET \textit{Overjet} > 5mm \\
    ALORS \textit{Diagnostic} = Classe II division 1 \parencite{thiruvenkatachari2017}.
    
    \item \textbf{Règle R-B3 (Masque Facial)} :
    SI \textit{Diagnostic} = Classe III squelettique ET \textit{Âge} entre 6 et 9 ans ET \textit{Rétromaxillie} = Vrai \\
    ALORS \textit{Appareil} = Masque de Delaire \parencite{owens2024}.
    
    \item \textbf{Règle R-C1 (Contre-indication Parodontale)} :
    SI \textit{Parodontite} = Active OU \textit{Perte d'attache} > 50\% \\
    ALORS \textit{Traitement} = Contre-indiqué \parencite{proffit2018}.
\end{itemize}

\subsection{Arbre de Déduction (Simplifié)}
Le raisonnement suit une logique en cascade :
\begin{enumerate}
    \item \textbf{Niveau 1 (Squelettique/Dentaire)} : Analyse des mesures céphalométriques (ANB, Wits) et cliniques $\rightarrow$ \textit{Diagnostic}.
    \item \textbf{Niveau 2 (Thérapeutique)} : Croisement du \textit{Diagnostic} avec l'\textit{Âge}, la \textit{Coopération} et la \textit{Croissance} $\rightarrow$ \textit{Appareil}.
    \item \textbf{Niveau 3 (Sécurité)} : Vérification des limites (Parodonte, limites techniques) $\rightarrow$ \textit{Contre-indications} ou \textit{Accessoires}.
\end{enumerate}

\section{Implémentation en Common Lisp}

\subsection{Représentation des Connaissances}
Nous avons opté pour l'utilisation de structures (`defstruct`) pour représenter les faits et les règles, offrant une meilleure lisibilité que les listes simples.

\begin{lstlisting}[caption=Structure des données en Lisp]
(defstruct fait
  attribut   ; Symbole (ex: 'age)
  valeur     ; Valeur (ex: 12)
  source     ; :utilisateur ou :deduit
)

(defstruct regle
  id          ; Identifiant
  premisses   ; Liste ((attribut operateur valeur)...)
  conclusions ; Liste ((attribut valeur)...)
  (active t)  ; Gestion de l'activation
)
\end{lstlisting}

\subsection{Moteur d'Inférence : Chaînage Avant}
Le choix s'est porté sur un moteur en **chaînage avant** (dirigé par les données). En diagnostic médical, nous partons des symptômes (faits observés) pour en déduire une pathologie (nouveaux faits). 

Le moteur fonctionne par saturation de la base de faits (`*base-faits*`).

\begin{lstlisting}[caption=Extrait du moteur de saturation]
(defun chainage-avant ()
  (loop while nouveau-fait-trouve do
    (dolist (r *base-regles*)
      (when (regle-active r)
        ;; Verification des premisses avec evaluer-condition
        (if (conditions-ok)
            (progn
               (ajouter-nouveaux-faits)
               (desactiver-regle r)))))))
\end{lstlisting}

Une particularité de notre implémentation est la fonction `evaluer-condition` qui gère des opérateurs de comparaison complexes (`>`, `<`, `member`, `equal`), dépassant le simple ordre 0 binaire.

\subsection{Fonctions de Service}
Le système inclut des fonctions robustes pour :
\begin{itemize}
    \item `ajouter-fait` : Vérifie l'unicité avant l'insertion.
    \item `charger-regles` : Initialise la base avec 15 règles métier.
    \item `reinitialiser-base` : Vide la mémoire de travail pour un nouveau patient.
\end{itemize}

\section{Utilisation de l'IA Générative}

Conformément aux consignes, l'IA (modèle Gemini) a été utilisée comme "Technicien Lisp" tandis que l'étudiant a agi comme "Expert Orthodontiste".

\subsection{Documentation des Prompts}
\begin{itemize}
    \item \textbf{Prompt initial} : Définition du rôle et contrainte "Attends ma base de règles".
    \item \textbf{Apport de l'IA} : 
        \begin{itemize}
            \item Structuration du code Lisp (séparation Modèle / Vue / Contrôleur).
            \item Écriture de la fonction récursive de chaînage arrière (`verifier-but`) pour comparaison.
            \item Génération des scénarios de tests unitaires (`test-edge-cases`).
        \end{itemize}
    \item \textbf{Vérification humaine} : Correction par l'étudiant de la logique des règles `R-A4` (valeurs ANB négatives pour la classe III) mal interprétées initialement par l'IA.
\end{itemize}

\section{Scénarios et Résultats}

\subsection{Scénario Clinique : Classe II Division 1}
**Entrées Utilisateur :**
\begin{itemize}
    \item Âge : 10 ans
    \item Relation Molaire : Classe 2
    \item Incisives Maxillaires : Proclinées
    \item Overjet : 6mm
    \item Coopération : Bonne
\end{itemize}

**Traces d'exécution :**
\begin{verbatim}
--- Cycle 1 ---
DÉCLENCHEMENT RÈGLE R-A2 : Classe II division 1 classique
[INFO] Nouveau fait établi : DIAGNOSTIC = CLASSE-2-DIV-1 (:deduit)

--- Cycle 2 ---
DÉCLENCHEMENT RÈGLE R-B1 : Twin Block ou Activateur
[INFO] Nouveau fait établi : APPAREIL = FONCTIONNEL (:deduit)
\end{verbatim}

**Résultat :** Le système recommande correctement un appareil fonctionnel, cohérent avec la littérature pour un patient en croissance \parencite{koretsi2015}.

\subsection{Tests de Robustesse}
Le système a été soumis à des tests limites (fonction `test-edge-cases` dans le code) :
\begin{itemize}
    \item Gestion des âges limites (passage enfant/ado/adulte).
    \item Données manquantes (le système répond "Diagnostic indéterminé" sans planter).
    \item Boucles infinies évitées par le flag `active` dans la structure `regle`.
\end{itemize}

\section{Conclusion et Perspectives}

Le système expert développé répond au cahier des charges d'un SE d'ordre 0+. Il permet de valider la logique diagnostique en orthodontie sur des cas standards.

\textbf{Limites :}
\begin{itemize}
    \item Rigidité des seuils (un overjet de 5.1mm déclenche la règle, 4.9mm non).
    \item Pas de gestion de l'incertitude (logique floue nécessaire pour les cas "borderline").
\end{itemize}

\textbf{Améliorations futures :}
\begin{itemize}
    \item Implémentation d'un SE d'ordre 1 (avec variables) pour généraliser les règles dentaires (gauche/droite).
    \item Interface graphique pour la saisie des données céphalométriques.
\end{itemize}

\printbibliography

\end{document}