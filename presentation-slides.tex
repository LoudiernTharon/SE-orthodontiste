\documentclass{beamer}
\usepackage[utf8]{inputenc}
\usepackage[french]{babel}
\usepackage{listings}
\usetheme{Madrid}
\usecolortheme{whale}

% Configuration Lisp pour les slides
\lstdefinelanguage{Lisp}{
  morekeywords={defstruct, defun, loop, when, push},
  basicstyle=\ttfamily\scriptsize,
  keywordstyle=\color{blue}\bfseries,
  commentstyle=\color{green!60!black},
  stringstyle=\color{orange},
}

\title{Système Expert d'Aide au Diagnostic Orthodontique}
\subtitle{TP02 - Intelligence Artificielle Symbolique}
\author{Loudiern Tharon \and Lou Aubert-Debrue}
\institute{Université de technologie de Compiègne}
\date{19 Décembre 2025}

\begin{document}

\begin{frame}
  \titlepage
\end{frame}

\begin{frame}{Plan de la présentation}
  \tableofcontents
\end{frame}

\section{Contexte et Objectifs}
\begin{frame}{Problématique Médicale}
  \textbf{Domaine :} Orthopédie Dento-Faciale (ODF).
  \vspace{0.5cm}
  
  \textbf{Le Problème :}
  \begin{itemize}
      \item Complexité du diagnostic (multiples paramètres : squelettique, dentaire, fonctionnel).
      \item Risque d'erreur dans le choix de l'appareillage (fixe vs fonctionnel vs chirurgie).
  \end{itemize}
  
  \vspace{0.5cm}
  \textbf{Notre Solution :}
  \begin{itemize}
      \item Système Expert d'Ordre 0+.
      \item Aide à la décision pour praticiens généralistes ou étudiants.
  \end{itemize}
\end{frame}

\section{Base de Connaissances}
\begin{frame}{Modélisation des Connaissances}
  Sources : Cochrane Reviews, HAS, Classification d'Angle.
  
  \begin{block}{Exemple de Règle Métier (R-A2)}
    \textbf{SI} Relation Molaire = Classe 2 \\
    \textbf{ET} Incisives = Proclinées \\
    \textbf{ET} Overjet > 5mm \\
    \textbf{ALORS} Diagnostic = Classe II Division 1
  \end{block}

  \begin{block}{Exemple de Règle Thérapeutique (R-B1)}
    \textbf{SI} Diagnostic = Classe II Div 1 \\
    \textbf{ET} 8 $\le$ Âge $\le$ 12 ans \\
    \textbf{ET} Coopération = Bonne \\
    \textbf{ALORS} Appareil = Fonctionnel (Activateur)
  \end{block}
\end{frame}

\section{Implémentation Technique}
\begin{frame}[fragile]{Choix Techniques en Common Lisp}
  \textbf{Représentation :} Structures (`defstruct`) pour une meilleure sémantique.
  
  \begin{lstlisting}[language=Lisp]
(defstruct fait
  attribut   ; ex: 'overjet
  valeur     ; ex: 6
  source)    ; :utilisateur ou :deduit

(defstruct regle
  id premisses conclusions active)
  \end{lstlisting}
  
  \vspace{0.5cm}
  \textbf{Moteur d'Inférence :}
  \begin{itemize}
      \item \textbf{Chaînage Avant} (Forward Chaining).
      \item Méthode par \textit{saturation} de la base de faits.
      \item Gestion des comparateurs : $>, <, \ge, =, \text{member}$.
  \end{itemize}
\end{frame}

\begin{frame}[fragile]{Logique du Moteur}
  \begin{lstlisting}[language=Lisp]
(defun chainage-avant ()
  (loop while nouveau-fait-trouve do
    (dolist (r *base-regles*)
      (when (regle-active r)
        (let ((ok (verifier-premisses r)))
          (when ok
            (declencher-regle r)
            (desactiver-regle r)))))))
  \end{lstlisting}
  
  \begin{alertblock}{Pourquoi le chaînage avant ?}
    En médecine, on part des symptômes (données) pour aller vers le diagnostic (but). C'est une approche "Data-Driven".
  \end{alertblock}
\end{frame}

\section{Démonstration et Résultats}
\begin{frame}{Scénario de Test : L'enfant "Classe II"}
  \textbf{Données Patient :}
  \begin{itemize}
      \item Âge : 10 ans
      \item Molaire : Classe 2, Overjet : 6mm
      \item Coopération : Bonne
  \end{itemize}

  \vspace{0.5cm}
  \textbf{Exécution du Système :}
  \begin{enumerate}
      \item \texttt{Cycle 1} : Règle R-A2 s'active $\to$ \textbf{Diag : Classe II Div 1}.
      \item \texttt{Cycle 2} : Règle R-B1 s'active (grâce au diag + âge) $\to$ \textbf{Appareil : Fonctionnel}.
      \item \texttt{Cycle 3} : Aucune nouvelle déduction. Arrêt.
  \end{enumerate}
  
  \textbf{Validation :} Cohérent avec les revues systématiques (Koretsi et al., 2015).
\end{frame}

\section{Collaboration IA}
\begin{frame}{Usage de l'IA Générative}
  \textbf{Rôles définis :}
  \begin{itemize}
      \item \textbf{Humain :} Expert du domaine (fournit les règles, valide la logique médicale).
      \item \textbf{IA (Gemini) :} Expert technique (implémentation Lisp, syntaxe, tests unitaires).
  \end{itemize}

  \vspace{0.5cm}
  \textbf{Apports clés :}
  \begin{itemize}
      \item Génération rapide de la structure du code (`boilerplate`).
      \item Création de jeux de tests automatisés (`test-edge-cases`).
      \item Débogage des parenthèses Lisp.
  \end{itemize}
\end{frame}

\section{Conclusion}
\begin{frame}{Conclusion et Perspectives}
  \textbf{Bilan :}
  \begin{itemize}
      \item Système fonctionnel et robuste (gestion des erreurs).
      \item Base de règles validée par la bibliographie.
      \item Code modulaire et documenté.
  \end{itemize}

  \vspace{0.5cm}
  \textbf{Limites (Ordre 0+) :}
  \begin{itemize}
      \item Effets de seuil (ex: âge strict à 12 ans).
      \item Manque de nuance (pas de "peut-être").
  \end{itemize}

  \vspace{0.5cm}
  \begin{center}
      \textit{Merci de votre attention.}
  \end{center}
\end{frame}

\end{document}
